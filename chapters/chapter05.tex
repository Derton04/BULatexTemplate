%!TEX root = ../main.tex

\chapter{Literature Review}
\label{chap:litReview}
The concept of paperless has been around for some time now and has long been the ambition for most organisations and individuals, however it's yet to be realised. Paper has become an integral part of our daily lives; people learnt how to properly organize their tasks, previous generations have used it to pass their wisdom to the current generation, many individuals cannot imagine a life without paper. 
However with the rapid growth of Information Technology(IT), many things are changing. Nowadays most people want immediate or instant services, they want access to information at any given time and place. As such the pedestal that paper used to rest on appears to be becoming obsolete, with an upward trend in the number of organizations trying to reduce the use of paper.

\cite{7760717} published an article on paperless Universities, they mention that despite the progression of IT, being paperless is still a dream for many companies and the case looks more critical when it comes to Universities - \say{the biggest paper consumer organization}.
They posed a pressing question, \say{Is it possible to make a campus paperless, or will it forever remain as a dream?}. Nowadays universities around the world have adopted computer-based systems. However, the management systems of universities are still paper based. The article goes on to mention some reasons for going paperless as the following;
\begin{itemize}
	\item Reduce the amount of paper used;
	\item Reduce the amount of time spent searching and retrieving documents;
	\item Reduce the amount of duplicate data;
	\item Decrease the physical space allocated to file storage;
	\item Increase flexibility of documents' use and security;
	\item Increase efficiency of university administration processes;
		\item help to save natural resources;
\end{itemize}

According to an article published on the Yale University website, in 2011, the University consumed 211,033 reams of paper - \say{that’s enough paper, if laid end to end, will stretch three-quarters of the way around the earth}. Using paper doesn’t only take up natural resources, it's also very expensive, (i.e, printing, storage, disposing, recycling and processing). Recognising this, Yale University's departments and schools initiated different steps to go green; not only are they benefiting the environment but gained financially by shifting to digital solutions. For example:
The office of Student Employment saved \$100,000 by replacing paper timesheets to electronic timetables.
The department of Finance and Business Operations managed to save around \$60,000 when they started publishing their annual reports online, \cite{Monday}.

The above showcases the reason in going paperless in going paperless:
Cost saving, improving data sharing, avoiding duplicate. 
Many organisations are trying to implement paperless, however, it is not an easy task. Although many people are comfortable with technology there are still reservations when new developments are made, this is unsurprising as most fear the unknown. A paperless world is a huge leap into the unknown. However this may be managed in two ways: Graphic User Interface(GUI) and Gamification.

As mentioned in \cite{7760717} article, a crucial factor in going paperless is simplified GUI. Developers tend to go overboard with the amount of features implemented and end up having a cramped system with lots of redundant information. As a result, users cannot highlight what is most important and become overwhelmed. Thus, developers should focus on making easy and concise GUI to make it more user-friendly for the target audience.

\textbf{Gamification - } is a comparatively new term that is used to describe the use of game elements and game design technique in non game context. The aim of this method is to induce a certain behaviour in people by relying on game mechanics to improve motivation and involvement in a specific task.
According to the self Determination Theory written by \cite{deci2008self}, people have extrinsic and intrinsic motivation to perform a task. Extrinsic is about the rewards and prizes, on the other hand intrinsic is related to the task performed because its fun and people like it. Websites like stakoverflow uses gamification to try and keep user motivated to perform a certain task.
This system is normally driven by points, for example on stackoverflow, an online programming community,  rewards users with points and badges after they have posted a number of questions or replied to questions posted by other users. The most popular elements are leaderboard and bagdes,

Nowadays, the newer generations make exhaustive use of digital technologies. They like to learn relying on technology, its not surprising that, the use of game elements in a non-game context seems a choice to be used in education, it has been used as an inspiration of engagement, motivation to boost learning by providing an environment that supports cooperation, competition, feedback and reward, \cite{kaplan2010users}.

\cite{hamari2014does} made a research about “Does Gamification Work?”, based on his assumption, gamification is seen in three main component. First is the motivation effect, second is the psychological outcomes and third is behavioral outcomes. Ways to entice users to keep using a gamification product is to make them feel that the activities are worthy. Applying gamification have to go through a clear arrangement to have an impact on users. The game elements has to be selected cautiously because every game element has it's own function. A bad gamification design can have an harmful effect on the system, \cite{hanus2015assessing}.

According to a survey made by \cite{uskov2014gamification}, in which they interviewed 18 volunteer students to collect their feedback about the impact of badges and leaderboards during a course experience and they also interviewed six students which were randomly selected for interviews about their perceptions about the course and the use of game elements. Their results shows that, the concept of badges had a greater impact on the motivation of students than the leaderboards. Students had shown more interest in gaining badges and saw them a social reward and secondary objectives to strive for in the course.

\section{Applying Gamification to Univent}
\label{gamification}
Gamification can be used as a means of promoting rewards for completing tasks. In the learning environment, students can be rewarded for taking the initiative to improve their soft skills. In this way, some of the discrepancies in personal efforts that are often present in student project work are reduced.
The same technique can be applied to Univent, allowing individuals to publish an event and rewards up front often goes a long way to ensuring that all individuals are working towards paperless. Moreover, integrating a gamification system according to which, when a user achieves a goal it triggers a reward system based on gaining points. In this way, the user’s interactions with the app will be more focused on going paperless. The reward system based on points is aimed to measure the paperless success of a host, the number will increase, and decrease based on the reaction of their events, so the points will change depending on the upvote or downvote that were made on the host's feed. The same effect on the points was determined by how many paperless events a user decides to host (host will be able to tick a box to declare they are going 100\% paperless for that event, meaning the will only use Univent to advertise  plus they will not use paper base advert such as posters). The more point and upvotes a host gets can get them more ad space, meaning that each event they post will be of high priority and will be added at the top of the event list and will further be added to the recommendation section on the app, potentially increasing the number of views.

\section{Related Work}
\label{similar_app}
This section will explore the functionality and the utility of existing solution that will be viewed as competitors.

\subsection{Fever}
Fever is an event discovery and booking app for iOS and Android. It provides information on the best events in major holiday destinations in Spain with expansion to major international cities like London, New York. Fever inspires users to find out about what these cities has to offer, from music festivals to fashion and restaurants.

The app also looks at each user’s top three interests, their activity, and interactions with other Fever users to offer a better recommendation engine. 

Fever is more a location based app, meaning the app uses the users location in order to find which events to promote. Although it's not targeted at the locals as they may already be aware of most of the services suggested, conversely it is likely to be of tremendous interest to people who are new to the city and will like to discover the best restaurants. 
Furthermore Fever only publishes event that they have personally selected and is not open to public posting. However, if someone outside of Fever wants to promote on their app they can fill out a form, but this is a long process to advertise on the app.
The app is only currently available in: New York, London, Paris, Madrid, Barcelona, Seville, Valencia, Bilbao and Malaga. 

\subsection{EventBrite}
This application allows the user to find what’s trending nearby at a wide variety of venues. They can keep up to date with any upcoming event like concerts, festivals, holiday events and networking events. Eventbrite enables users to buy tickets directly on the app stored on the mobile phone for convenience access. Users can also store their credit and debit card details for faster payment.
The application can find things to do based on what the user is into, where they want to go or when they want to go out. This will provide a recommendation of events for the user, based on all these information. The application includes a share feature where users can share events with friends and vice versa and the app is available in German, Spanish, French, Italian, Dutch, Portuguese, and Swedish.

Overall the Eventbrite app has similar features to what is described in the application vision, see section \ref{app vision} and even more features and currently more than 5 million downloads according to the play store. One major drawback of this application, however, is that, the mobile app doesn't have the feature that allows to add events. The only way to add an event is to use their main website and it's more suited to larger enterprises who are looking for a more convenient, visual way to manage their events. 

Currently Eventbrite is based on events worldwide, giving the user access to events in various parts of the world. Univent will still be unique, as at the moment there is no app that covers events or activities just for Universities (in this case the Bournemouth University) or any event related to the members enrolled on the faculty.

\section{Summary}
Eventbrite and Fever are good examples of how Univent will work. They showcases the main features and how tasks can be handled.
Furthermore, the review of both provides a better understanding of what features are required and gives the opportunity to identify any missing features that could be important to implement for an all round user experience.


