%!TEX root = ../main.tex
\chapter{Evaluation}
\section{Overview}
As usability is quite crucial for the success of the application, a random  selection of users were offered to try the application. Once they had tested the app they were asked to leave a review on the play store and fill out a survey.
Below is the outcome of the survey and some of the reviews.

\begin{longtable}{| L{7cm} | L{7cm} |}
	Question & Score  \\ \hline
	How would you rate the mobile app? & Weighted average is 4.75 out of 5 stars \\	\hline
	I found the system really easy to use &	75\% Strongly agree, 25\% Agree  \\ \hline
	I would need assistance to be able to use the app.   &	50\% neither agree or disagree, 25\% diasagree and the other 25\% strongly disagree  \\ \hline
How visually appealing is the app?  &50\% Extremely appealing, 25\% Very appealing and the other 25\% Somewhat appealing \\	\hline
I think I would use the app frequently  & 75\% agree and 25\% strongly agree \\	\hline

How likely is it that you would recommend the app to a friend or colleague?  & 50\% Extremely likely and 50\% very likely\\	\hline

Do you have any other comments about how we can improve the app?  &	see table \ref{comments}\\	\hline	\caption{Survey}
	\label{survey}
\end{longtable}
\pagebreak
\begin{longtable}{| L{6cm} | L{6cm} |}
	User & Comments  \\ \hline
	User 1  &	Good app, allows you to view events from Bournemouth University, allows you to add your own events, simple to use nice layout\\	\hline
		User 2  &	Very good app, simple user friendly interface \\	\hline
		User3  &	Create a version compatible with apple iPhones  \\	\hline
			User 4  &	Some usability issues when you verify your email address, I had to re enter Bournemouth University before i could sign in, I didnt know I had to do that without guidance  \\	\hline
			User 5  &	Pretty good app. Perfect for checking available BU uni events - convenient use. Layout is nice and simple. Good simple app \\	\hline
		\caption{User Comments}
		\label{comments}
\end{longtable}	

For general feedback, refer to table \ref{survey} and \ref{comments}. The screen design received very good feedback; the participants noticed how the screens were appealing and they also noticed how useful the artefact could be.

\subsection{Requirements Evaluation}
In terms of functionality all the requirements mentioned in chapter \ref{req_analysis} have been met.

\begin{longtable}{|  L{10cm} | c |}
	\hline
	  Requirements\newline & Met? 
	  
	  \newline Yes/No
	   \\	
	\hline
	 The application should have a register and login screen & Yes  \\
	\hline
	 The application should integrate a validation for email, only \newline Universities email are accepted & Yes  \\
	\hline
	 The application should provide specific information such as event location, event start and end date, description, title. & Yes  \\
	\hline
	 The application should provide information of all attendees interested in an activity or event  & Yes \\
	\hline
	 The application should have a \say{Share} feature. This will open any related app installed on the users phone so they can share the event with friends & Yes  \\
	\hline
	 The application should provide a \say{Send Email} feature. This will open any email client installed on users phone, so they can email the event host for further information regarding the event & Yes  \\
	\hline
	 The application should have a \say{Map Feature}. This will direct the user to the Maps application with the event location already pre entered.  & Yes \\
	\hline
	 The application should have an \say{Add to Calendar} feature. This will open any calender application on the users phone and it will allow one to save the event on the calender creating a future reminder. & Yes  \\
	\hline
	 The application should allow the user to choose the category they are most interested in and based on that information display all types of event or activities in those categories & Yes \\
	\hline
	  The application should save the user login details, so user is only required to login once. & Yes \\
	\hline
	The application should show the location on a map directly on the screen of the app  & Yes \\
	\hline
	%	\end{tabular}}
	\caption{Requirements Evaluation}
	\label{table:req_eval}
\end{longtable}	

\section{Success Criteria}
The aim of this thesis was to provide support in publishing and keeping track with activities in the faculty.
The success criteria set in section \ref{success_criteria}, has been successfully met and all requirements have been implemented.

The app was built using Android studio, the Android framework and API were discussed in chapter \ref{background}, that knowledge aid in implementing the app successfully. 
 
Overall this project was successful, all feedback regarding the system received from various users indicates users were able to use the app successfully and were able to create an event as well as keep track of any other activities posted on the app.