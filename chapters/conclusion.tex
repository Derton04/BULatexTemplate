%!TEX root = ../main.tex
\chapter{Conclusion}
\section{Summary}
The artefact delivers exactly what it is was built for, which was to build an android app that encourages users to use the app as a platform to share events. A server side was created as a way to communicate between server and client, on the client side HttpUrlConnection and Volley were used to retrieve and send data to the server. Both solutions were adopted because of their efficiency  and simplicity in being implemented, ignoring other existing solutions like okhttp and retrofit.

The app allows users to publish their own events and view any other upcoming events on the campus. This makes publishing very easy for users, and individuals can find all information needed about an event on the go. 

The feedback received from users was very positive, and majority of them will definitely consider using the App in their daily life. 

This project will contribute in going paperless. Which will ensure cost savings for the University and will enable efficient communication and collaboration among individuals. Univent will definitely make an impact. For Universities to be efficient they have to adopt similar solutions which align with environmental friendliness. 

The App met all requirements, however during testing and evaluation, further ideas were introduced, which will be implemented depending on future funding.

\section{Future Works}
Building an app, is a very complex task and time consuming process.
Apps need to be improved and updated constantly to entice users. Even if the main requirement were met and the app has been published, there are always room for improvement.

As mentioned in section \ref{the_challenge}, integrating the mobile application with the University, will depict that the app is useful and can make a change to how events and activities are published and accessed in the university. This is definitely a challenge, however it will be a great achievement if it was possible to work alongside the university.

Future implementation could be:

\textbf{Database Security - }
Most of the information of the user is 
stored in the backend databases. One of the main vulnerabilities is SQL (Structured Query Language) injection attack. SQL injection attack is one of the main vulnerabilities and prevalent database attacks. The attacker,
by exploiting the application and database she/he can get unauthorized access to the 
database and cause harm. A solution to this problem will be to use prepare statement in the server side before 
executing the query.

\textbf{Include Gamification - }
Including a gamification system to the app might motivate users to use the mobile app to publish events an go greener, refer to section \ref{gamification} about gamification. 

\textbf{Push Notification - }
push notifications will remind users about any upcoming event they are interested in and also remind them of any update about any event they are attending. One good side of having push notification is that it reminds users of your app and improving the chances of the app remaining installed on their devices. Google Cloud Messaging(GCM), is a free service used to send push notifications to users and ensures notifications are delivered securely and reliably.

\textbf{Follow Friends - }
Users should be able to follow friends and also get event notification on what their friends are interested in. 

\textbf{Keep User Updated - }
Users should be notified when they decide to attend an event; a list of all essential information about that event should be sent to the users email and the users should be kept updated about any changes on the event. For example if the event is cancelled for any reason the user should be notified about the cancellation.

\textbf{Extend to iOS - }
When most of the features have been implemented on the android platform, it is essential to have a stable app on one platform before implementing it on another, to avoid any drawbacks, the app can finally be extended to the iOS platform to reach more users.

\textbf{\textit{Word Count: 9940}}
